
% "Лаба"

\documentclass[a5paper,10pt, twoside]{article} % тип документа

\usepackage{hyperref}
\usepackage{fancyhdr}
\usepackage{import}

% Математика

\import{../../headers/}{math.tex}

%  Русский язык

\import{../../headers/}{russian.tex}

% Дефайны

\import{../../headers/}{my_defs.tex}

%\fancyhf{}
\renewcommand{\footrulewidth}{ .0em }
\fancyfoot[C]{\texttt{\textemdash~\thepage~\textemdash}}
\fancyhead[L]{Работа 4.1.2 \hfil}
\fancyhead[R]{\hfil Хайдари Фарид, группа Б01-901}

\pagestyle{fancy}

\graphicspath{{src/pics/}} % где лежат картинки

\counterwithin{figure}{section}

% Title Page
\title
{
\hfill \break	\hfill \break
\hfill \break	\hfill \break
Лабораторная работа 4.1.2.

МОДЕЛИРОВАНИЕ ОПТИЧЕСКИХ ПРИБОРОВ
}
\author{Хайдари Фарид, Б01-901}

%\setcounter{secnumdepth}{0}

\begin{document}

\maketitle


\thispagestyle{empty} % выключаем отображение номера для этой страницы

\newpage

\tableofcontents % Вывод содержания
\thispagestyle{plain}
\newpage


\paragraph{Цель работы:}

изучить модели зрительных труб (астрономической трубы Кеплера и земной трубы Галилея) и микроскопа, 
определить их увеличения.

\paragraph{В работе используются:}

оптическая скамья, набор линз, экран, осветитель со шкалой, зрительная труба, диафрагма, линейка.

\section{Теоретические сведения}
\import{src/}{1_theor.tex}

  \subsection{Увеличение астрономической зрительной трубы}
  \import{src/}{2_astr.tex}

  \subsection{Увеличение галилеевой зрительной трубы}
  \import{src/}{3_gal.tex}

  \subsection{Увеличение микроскопа}
  \import{src/}{4_micro.tex}

\section{Экспериментальная установка}
\import{src/}{5_ust.tex}

\section{Ход работы}
\import{src/}{6_hod_F.tex}


\end{document}