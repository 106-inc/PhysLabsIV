% "Лаба"

\documentclass[a5paper,10pt, twoside]{article} % тип документа

\usepackage{import}


%  Русский язык

\import{../../headers/}{russian.tex}

% Математика

\import{../../headers/}{math.tex}

% Дефайны

\import{../../headers/}{my_defs.tex}


\graphicspath{{pics/}} % где лежат картинки

% Title Page
\title
{
	\hfill \break	\hfill \break
	\hfill \break	\hfill \break
	Лабораторная работа 4.3.1.
	
	ПРЕОБРАЗОВАНИЕ ФУРЬЕ В ОПТИКЕ
}
\author{Хайдари Фарид, Б01-901}


\begin{document}
	
\maketitle


\thispagestyle{empty} % выключаем отображение номера для этой страницы

\newpage

\tableofcontents % Вывод содержания

\newpage


\paragraph{Цель работы:}

	исследование особенностей применения пространственного преобразования Фурье
	для анализа дифракционных явлений.


\paragraph{В работе используются:}

	гелий-неоновый лазер, кассета с набором
	сеток разного периода, щель с микрометрическим винтом, линзы,
	экран, линейка.

\section{Теоретические сведения}

\section{Экспериментальная установка}
	


\section{Ход работы}

\end{document} % конец документа
