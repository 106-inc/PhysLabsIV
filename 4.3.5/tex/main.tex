% "Лаба"

\documentclass[a5paper,10pt, twoside]{article} % тип документа

\usepackage{hyperref}

\usepackage{import}

% Математика

\import{../../headers/}{math.tex}

%  Русский язык

\import{../../headers/}{russian.tex}

% Дефайны

\import{../../headers/}{my_defs.tex}

\graphicspath{{pics/}} % где лежат картинки

% Title Page
\title
{
\hfill \break	\hfill \break
\hfill \break	\hfill \break
Лабораторная работа 4.3.4.

ПРЕОБРАЗОВАНИЕ ФУРЬЕ В ОПТИКЕ
}
\author{SOMEBEODY, Б01-901}

%\setcounter{secnumdepth}{0}

\begin{document}

\maketitle


\thispagestyle{empty} % выключаем отображение номера для этой страницы

\newpage

\tableofcontents % Вывод содержания

\newpage


\paragraph{Цель работы:}

исследование особенностей применения пространственного преобразования Фурье
для анализа дифракционных явлений.


\paragraph{В работе используются:}

гелий-неоновый лазер, кассета с набором
сеток разного периода, щель с микрометрическим винтом, линзы,
экран, линейка.

\section{Теоретические сведения}

  \subsection{Принципы голографии}
  \import{src/}{3.tex}

    \subsubsection{Голограмма точечного источника (голограмма Габора)}
    \import{src/}{3.1.tex}

\end{document}