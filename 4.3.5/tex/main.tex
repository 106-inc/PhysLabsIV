
% "Лаба"

\documentclass[a5paper,10pt, twoside]{article} % тип документа

\usepackage{hyperref}
\usepackage{fancyhdr}
\usepackage{import}

% Математика

\import{../../headers/}{math.tex}

%  Русский язык

\import{../../headers/}{russian.tex}

% Дефайны

\import{../../headers/}{my_defs.tex}

%\fancyhf{}
\renewcommand{\footrulewidth}{ .0em }
\fancyfoot[C]{\texttt{\textemdash~\thepage~\textemdash}}
\fancyhead[L]{Работа 4.3.5 \hfil}
\fancyhead[R]{\hfil SOMEBEODY, Б01-901 группа }

\pagestyle{fancy}

\graphicspath{{src/pics/}} % где лежат картинки

\counterwithin{figure}{section}

% Title Page
\title
{
\hfill \break	\hfill \break
\hfill \break	\hfill \break
Лабораторная работа 4.3.5.

ИЗУЧЕНИЕ ГОЛОГРАММЫ
}
\author{SOMEBEODY, Б01-901}

%\setcounter{secnumdepth}{0}

\begin{document}

\maketitle


\thispagestyle{empty} % выключаем отображение номера для этой страницы

\newpage

\tableofcontents % Вывод содержания
\thispagestyle{plain}
\newpage


\paragraph{Цель работы:}

Изучить свойства голограмм точечного источника и объемного предмета.

\paragraph{В работе используются:}

гелий-неоновыый лазер, голограммы, набор линз, предметная шкала, экран, линейка.

\section{Теоретические сведения}

  \subsection{Принципы голографии}
  \import{src/}{3.tex}

    \subsubsection{Голограмма точечного источника (голограмма Габора)}
    \import{src/}{3.1.tex}

      \paragraph{Зонная решётка Габора.}
      \import{src/}{3.1.1.tex}

      \paragraph{Сферическая восстанавливающая волна.}
      \import{src/}{3.1.2.tex}
    
    \subsubsection{Разрешающая способность голограммы}
    \import{src/}{3.2.tex}

    \subsection{Схема с наклонным опорным пучком}
    \import{src/}{3.3.tex}

\end{document}