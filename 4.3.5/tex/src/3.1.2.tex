Рассмотрим ещё одно свойство голограммы точки.
Пусть восстанавливающая волна является не плоской, а сферической,
расходящейся от точки, расположенной на расстоянии $z_1 > z_0$ от голограммы.
Используя параболическое приближение \eqref{eq::3.86}, запишем комплексную амплитуду падающей волны в
плоскости голограммы:
\begin{equation}\label{eq::3.89}
  f_- (\rho) = \e^{ -\iu \frac{k}{2z_1}\rho^2},
\end{equation}

Восстановленное поле в плоскости $z = 0$:
$$
f_+(\rho) = f_-(\rho) t(\rho).
$$

Используя \eqref{eq::3.89} и \eqref{eq::3.83}, получим (с точностью до несущественных констант)

$$
f_+(\rho) = \e^{ -\iu \frac{k}{2z_1}\rho^2} + 
\e^{ -\iu \frac{k}{2}\left( \frac{1}{z_1} + \frac{1}{z_0} \right)\rho^2} + 
\e^{ -\iu \frac{k}{2}\left( \frac{1}{z_1} - \frac{1}{z_0} \right)\rho^2}.
$$

Из этого выражения видно, что расстояние $b_2$ от голограммы до мнимого изображения источника уменьшается, так как
$$
\frac{1}{b_2} = \frac{1}{z_0} + \frac{1}{z_1},
$$

т.~е. это изображение приближается к голограмме.
В то же время расстояние $b_3$ от голограммы до действительного изображения увеличивается:
$$
  \frac{1}{b_3} = \frac{1}{z_0} - \frac{1}{z_1}.
$$

Это свойство голограммы следует иметь в виду при объяснении изменения масштабов изображений трёхмерного предмета при освещении голограммы сферической волной.

В случае произвольного предмета на голограмме записывается сложный интерференционный узор, который можно рассматривать
как совокупность зонных кольцевых решёток Габора, причём каждая
из них ответственна за восстановление \quot{своей} точки предмета на
стадии реконструкции изображения.
