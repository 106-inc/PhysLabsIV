
Рассмотрим более детально волну, ответственную за создание действительного изображения. Конечно, 
сходящаяся сферическая волна не фокусируется строго в точку $S''$ : изображением является не точка, 
а маленькое дифракционное пятнышко. Действительно, плоская восстанавливающая волна, пройдя сквозь 
голограмму, преобразуется в сферическую волну (мы говорим сейчас о той части волны, которая
ответственна за создание действительного изображения), причём сферический волновой фронт ограничен 
размерами голограммы. Эта сходящаяся сферическая волна (третье слагаемое в \eqref{eq::3.85})
не отличается от сферической волны, возникающей за собирающейся линзой, размер $D$ которой равен 
размеру голограммы. При фокусировке света линзой возникает дифракционное пятно Эйри, размер которого 
определяется формулой \eqref{eq::3.74}. Ясно, что и голограмма создаёт изображение -- дифракционное 
пятно, размер которого определяется аналогичной формулой:

\begin{equation}
    \Delta x \thicksim \frac{\lambda}{D} z_0
\end{equation}
\\
где $z_0$ — расстояние от точечного источника до голограммы в процессе записи, $D$ — размер голограммы.
Если записывать на голограмму изображения двух точек, расстояние между которыми меньше $\Delta x$,
то при восстановлении изображений возникают два пятна, налагающиеся друг на друга так, что, согласно
критерию Рэлея, они оказываются неразрешимыми. Сказанное касается, разумеется, и мнимого изображения
точечного источника: наблюдателю оно кажется маленьким пятнышком размера $\Delta x$, находящимся
на расстоянии $z_0$ за голограммой.

Важнейшее свойство голограммы состоит в том, что любой её малый участок содержит информацию обо всем 
объекте: ведь поле в каждой точке голограммы является суперпозицией полей, посылаемых всеми
точками предмета (и опорной волной). Другими словами, интерференционная картина на каждом небольшом 
участке голограммы содержит информацию об амплитуде и фазе колебаний всех точек предмета, поэтому 
изображение может быть восстановлено с помощью небольшого осколка голограммы, полученной при записи.
Разумеется, разрешающая способность определяется размером осколка.