
\subsection{Изучение характеристик голограммы точечного источника}

    В работе предлагается рассчитать расстояние от голограммы до точечного источника, 
    который использовался при её создании:
    \begin{itemize}
        \item по результатам измерения радиусов голографических колец,
              спроецированных на экран при помощи короткофокусной линзы
        \item по результатам измерения параметров проекционной установки, 
              в которой голограмма используется как фокусирующая линза, 
              а объектом служит предметная шкала.
    \end{itemize}

    \subsubsection*{Определение цены деления предметной шкалы}

        Устанавливаем кассету с транспарантами вблизи лазера. Освещаем лучом лазера шкалу и 
        получаем на удалённом экране дифракционную картину, созданную крестообразной шкалой.

        По результатам измерений $\Delta x = (0.55 \pm 0.01)$ см, $L = (118.5 \pm 0.2)$ см.
        $\lambda$ известно и равно $532$ нм. Отсюда из формулы для дифракции Фраунгофера
        находим цену деления: $D = \frac{\lambda \cdot L}{\Delta x} = (115 \pm 2)$ мкм.

        Определим цену деления той же шкалы, используя линзу с фокусным расстоянием
        $F = 40$ мм, $a = (4.6 \pm 0.1)$ см, $b = (113.5 \pm 0.1)$ см. Находим
        $\Gamma = \frac{b}{a} = (26.7 \pm 0.5)$. По результатам измерений
        $D' = (0.26 \pm 0.02)$ см. Таким образом $D = \frac{D'}{\Gamma} = (97 \pm 17)$

        По размерам погрешностей видно, что первый способ точнее.

    \subsubsection*{Определение расстояние от голограммы до точечного источника}

    \subsubsection*{Изучение фокусирующих свойств голограммы}

\subsection{Изучение характеристик голограммы объемного предмета}

    \subsubsection*{Изучение мнимого изображения}

    \subsubsection*{Изучение действительного изображения}

