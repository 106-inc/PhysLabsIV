
В работе исследуются голограммы точечного источника и трехмерного объекта.
Эти голограммы записаны на фотопластинках типа ЛОИ-2, имеющих высокое разрешение:
$\approx 5000$ линий/мм. При записи был использован He--Ne-лазер с длиной волны излучения 
$\lambda = 632,8$ нм. Такой же лазер используется при восстановлении изображения.

В работе требуется определить расстояние $d$ между голограммой и точечным источником
света. Это расстояние можно рассчитать по формуле \eqref{eq::3.88}, если измерить радиусы
$\rho$ нескольких колец в  нашей голограмме слишком малы для непосредственных измерений
($d \approx 1$ см, $\lambda \approx 0.6$ нм, $\rho $), поэтому требуется получить увеличенное
изображение голограммы. Это изображение получается при помощи короткофокусной линзы на
экране (рис. \ref{img::ust}).

Согласно формуле \eqref{eq::3.85} при просвечивании голограммы $\Gamma$ (рис. \ref{img::ust})
точечного источника плоской волной с амплитудой $f_0 = const$ на выходе имеется три волны: 
плоская с амплитудой $f_1 = const$, расходящаяся сферическая волна $f_2 \propto  \e^{ikr}$, 
отвечающая мнимому изображению $O_2$, и сходящаяся сферическая волна $f_3 \propto \e^{-ikr}$,
отвечающая действительному изображению $O_3$ . После прохождения линзы Л волна $f_1$ 
собирается в фокусе линзы в точке $O_1'$, волны $f_2$ и $f_3$ фокусируются соответственно 
в точках $O_2'$ и $O_3'$. Изображение, возникающее на экране Э, можно рассматривать как 
результат интерференции сферических волн от трёх точечных источников: $O_1'$ , $O_2'$ и $O_3'$.
Поэтому картина концентрических колец возникает на экране не только тогда, когда на нём 
образуется изображение $\Gamma'$ голограммы, но и при многих других положениях линзы. 
Это легко проверить экспериментально.

Таким образом, чтобы определить радиусы колец, следует убедиться, что на экране действительно 
возникло изображение голограммы, то есть что плоскости $\Gamma$ и $\Gamma'$ сопряжены. 
Для этого в плоскость $\Gamma$ вместо голограммы помещают прозрачную предметную шкалу, на 
которую нанесён тонкий крест с делениями. В этом случае увеличенное изображение креста 
получается при единственном положении линзы для определённого расстояния между предметом 
и экраном.

Предметная шкала и голограмма точечного источника смонтированы на платформе в единую плоскую 
кассету транспарантов (шкала -- в отверстии \textnumero 1, голограмма -- в отверстии 
\textnumero 2). Такая же предметная шкала, закреплённая в отдельной оправе, используется в 
качестве предмета при определении фокусного расстояния голографической линзы.

Кроме голограммы точечного источника в работе исследуется голограмма объёмного предмета. 
Предмет представляет собой горизонтально расположенную миллиметровую линейку, за которой 
расположен вертикальный металлический стержень. При записи голограммы предмет располагался 
на расстоянии $10$ см от пластинки. Фотопластинка была обращена к предмету той стороной, 
на которую нанесена эмульсия. Голограмма установлена в оправе вертикально и может 
поворачиваться вокруг вертикальной оси. Для измерения угла поворота служит
транспортир, закреплённый под голограммой в горизонтальной плоскости.

Источником света служит He--Ne-лазер с диаметром луча $< 1$мм. В опытах с объёмной 
голограммой требуются более широкие световые пучки. Для расширения пучка используются две 
линзы: короткофокусная -- с фокусным расстоянием $< 2$мм и длиннофокусная -- с фокусным 
расстоянием $\approx 8$см. Собранный из этих линз расширитель создаёт пучок диаметром $4-5$ см.

При проведении опытов оптические элементы размещаются на массивном оптическом столе и могут 
перемещаться вдоль оптической оси и в перпендикулярной ей плоскости. Расстояния измеряются 
линейкой. Фокусные расстояния линз указаны на их оправах.
