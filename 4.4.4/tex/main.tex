
% "Лаба"

\documentclass[a5paper,10pt, twoside]{article} % тип документа

\usepackage{hyperref}
\usepackage{fancyhdr}
\usepackage{import}

% Математика

\import{../../headers/}{math.tex}

%  Русский язык

\import{../../headers/}{russian.tex}

% Дефайны

\import{../../headers/}{my_defs.tex}

%\fancyhf{}
\renewcommand{\footrulewidth}{ .0em }
\fancyfoot[C]{\texttt{\textemdash~\thepage~\textemdash}}
\fancyhead[L]{Работа 4.4.4 \hfil}
\fancyhead[R]{\hfil Державин Андрей, группа Б01-901}

\pagestyle{fancy}

\graphicspath{{src/pics/}} % где лежат картинки

\counterwithin{figure}{section}

% Title Page
\title
{
\hfill \break	\hfill \break
\hfill \break	\hfill \break
Лабораторная работа 4.4.4.

ИНТЕРФЕРОМЕТР ФАБРИ-ПЕРО
}
\author{Державин Андрей, Б01-901}

%\setcounter{secnumdepth}{0}

\begin{document}

\maketitle


\thispagestyle{empty} % выключаем отображение номера для этой страницы

\newpage

\tableofcontents % Вывод содержания
\thispagestyle{plain}
\newpage


\paragraph{Цель работы:}

Изучение интерферометра Фабри-Перо и определение его характеристик, как спектрального прибора.

\paragraph{В работе используются:}

Интерферометры Фабри-Перо, линзы, светофильтр, ртутная лампа ПРК-2, высокочастотная натриевая лампа,
катетометры КМ-6.

\section{Теоретические сведения}
\import{src/}{1_theor.tex}

  \subsection{Измерение длин волн $\lambda$ и расстояний $d \lambda$ между спектральными линиями}
  \import{src/}{2_len.tex}

  \subsection{Дисперсия интерферометра}
  \import{src/}{3_disp.tex}

  \subsection{Дисперсионная область}
  \import{src/}{4_disp_area.tex}

  \subsection{Разрешающая способность интерферометра Фабри-Перо}
  \import{src/}{5_razr.tex}

\section{Экспериментальная установка}
\import{src/}{6_ust.tex}

\newpage 
\section{Ход работы}
\import{src/}{7_hod_A.tex}

\end{document}