Отношение расстояния $dl$ между спектральными линиями в плоскости спектра
к разности длин волн $d\lambda$ этих линий называют линейной дисперсией
$D^*$ спектрального прибора ($D^* = \der{l}{\lambda}$) и выражают 
обычно в миллиметрах на ангстрем. Можно выразить линейную дисперсию
$D^*$ через угловую ($\der{\Theta}{\lambda}$). 
Как следует из формулы \eqref{eq::9}, для интерферометра Фабри-Перо
$$
D^* = f \der{\Theta}{\lambda} = 
\frac{dD}{2d\lambda} = \frac{2f^2}{\lambda D}.
$$
\noindent Высокая дисперсия является основным преимуществом интерферометра Фабри–Перо.