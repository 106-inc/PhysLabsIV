Областью дисперсии спектрального прибора называют максимальный 
интервал длин волн $\Delta\lambda$, при котором ещё не происходит
перекрытия интерференционных полос соседних порядков. 
Ширина этой области определяется из условия наложения кольца
($m+1$)-го порядка для длины волны $\lambda$ и кольца $m$-го 
порядка для длины волны $\lambda+\Delta\lambda$:
$$
m\left(\lambda+\Delta\lambda\right)=\left(m+1\right)\lambda,
$$
\noindent откуда
\begin{equation}\label{eq::10}
  \Delta\lambda = \frac{\lambda}{m} \approx \frac{\lambda^{2}}{2L}.
\end{equation}

Порядок интерференции $m$ в интерферометрах Фабри-Перо чрезвычайно высок. Так, для 
$\lambda = 5 \cdot 10^{-5}$ см и $L = 0.5$ см получаем:
$m \approx 2 L / \lambda = 2 \cdot 10^4$. Область дисперсии при этом равна
$\Delta \lambda = 0.25$ \AA. Таким образом, спектральный интервал, который можно 
анализировать с помощью интерферометра Фабри-Перо, весьма мал. Поэтому перед
интерферометром Фабри-Перо обычно располагают светофильтр или другой
спектральный прибор, вырезающий спектральную полосу, не превышающую $\Delta \lambda$.
Отметим, что если спектральная полоса $\Delta \lambda$ исследуемого излучения
 известна, то с помощью формулы \eqref{eq::10} можно определить допустимое 
 значение постоянной интерферометра $L$.

\noindent 