Областью дисперсии спектрального прибора называют максимальный 
интервал длин волн $\Delta\lambda$, при котором ещё не происходит
перекрытия интерференционных полос соседних порядков. 
Ширина этой области определяется из условия наложения кольца
($m+1$)-го порядка для длины волны $\lambda$ и кольца $m$-го 
порядка для длины волны $\lambda+\Delta\lambda$:
$$
m\left(\lambda+\Delta\lambda\right)=\left(m+1\right)\lambda,
$$
\noindent откуда
\begin{equation}\label{eq::10}
  \Delta\lambda = \frac{\lambda}{m} \approx \frac{\lambda^{2}}{2L}.
\end{equation}
\noindent 