Разрешающая способность спектрального прибора определяется отношением
\
\begin{equation}
  R = \frac{\lambda}{\delta \lambda}
\end{equation}
\
где $\delta \lambda$ -- минимальная разность длин волн, разрешимая 
прибором вблизи длины волны $\lambda$. При определении $\delta \lambda$
обычно используют условный критерий разрешения Релея, согласно 
которому две линии разрешаются, если их максимумы отстоят друг от 
друга на половину их ширины. Определяя ширину линии на уровне, на 
котором интенсивность падает в два раза по сравнению с максимальным 
значением в середине линии, можно получить из \eqref{eq::4}:
\
\begin{equation}\label{eq::12}
  R \approx \frac{2 \pi L \sqrt{r}}{\lambda (1 - r)} \approx \frac{2 \pi L}{\lambda (1 - r)}
\end{equation}
\
Из \eqref{eq::12} следует, что при $r \rightarrow 1$ добротность
$Q \approx R \rightarrow \infty$. Однако на самом деле этого не происходит.
При $r$, достаточно близких к единице, существенное влияние на
разрешающую способность начинает оказывать рассеяние света из-за
неоднородностей поверхностей зеркал. Например, при
$r \approx 95\%$ глубина неровностей поверхности должна быть
меньше $\lambda / 50$
