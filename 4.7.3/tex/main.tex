\documentclass[a5paper,10pt, twoside]{article} % тип документа

\usepackage{hyperref}
\usepackage{fancyhdr}
\usepackage{import}

% Математика

\import{../../headers/}{math.tex}

%  Русский язык

\import{../../headers/}{russian.tex}

% Дефайны

\import{../../headers/}{my_defs.tex}

%\fancyhf{}
\renewcommand{\footrulewidth}{ .0em }
\fancyfoot[C]{\texttt{\textemdash~\thepage~\textemdash}}
\fancyhead[L]{Работа 4.3.5 \hfil}
\fancyhead[R]{\hfil SOMEBEODY, Б01-901 группа }

\pagestyle{fancy}

\graphicspath{{src/pics/}} % где лежат картинки

\counterwithin{figure}{section}

% Title Page
\title
{
\hfill \break	\hfill \break
\hfill \break	\hfill \break
Лабораторная работа 4.3.5.

ИЗУЧЕНИЕ ГОЛОГРАММЫ
}
\author{SOMEBEODY, Б01-901}

%\setcounter{secnumdepth}{0}

\begin{document}

\maketitle


\thispagestyle{empty} % выключаем отображение номера для этой страницы

\newpage

\tableofcontents % Вывод содержания
\thispagestyle{plain}
\newpage


\paragraph{Цель работы:}

ознакомление с методами получения и анализа поляризованного света.

\paragraph{В работе используются:}

оптическая скамья с осветителем; зеленый светофильтр; два поляроида; черное зеркало; 
полированная эбонитовая пластинка; стопа стеклянных пластинок; слюдяные пластинки 
разной толщины; пластинки в $1/4$ и $1/2$ длины волны; пластинка в одну длину волны для 
зеленого света (пластинка чувствительного оттенка)

\section{Теоретические сведения}

  \subsection{Естественный и поляризованный свет}
  \import{src/}{1_1.tex}

  \subsection{Методы получения линейно поляризованного света}
  \import{src/}{1_2.tex}

  \subsection{Определение направления разрешенной плоскости колебаний поляроида.}
  \import{src/}{1_3.tex}

  \subsection{Получение эллиптически поляризованного света}
  \import{src/}{1_4.tex}

  \subsection{Анализ эллиптически поляризованного света}
  \import{src/}{1_5.tex}

  \subsection{Пластинка чувствительного оттенка}
  \import{src/}{1_6.tex}

  \subsection{Интерференция поляризованных лучей}
  \import{src/}{1_7.tex}


Ход работы
\section{}
\import{src/}{hod_F.tex}

\end{document}