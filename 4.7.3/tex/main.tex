\documentclass[a5paper,10pt, twoside]{article} % тип документа

\usepackage{hyperref}
\usepackage{fancyhdr}
\usepackage{import}

% Математика

\import{../../headers/}{math.tex}

%  Русский язык

\import{../../headers/}{russian.tex}

% Дефайны

\import{../../headers/}{my_defs.tex}

%\fancyhf{}
\renewcommand{\footrulewidth}{ .0em }
\fancyfoot[C]{\texttt{\textemdash~\thepage~\textemdash}}
\fancyhead[L]{Работа 4.7.3 \hfil}
\fancyhead[R]{\hfil Андрей Державин, Б01-901 группа }

\pagestyle{fancy}

\graphicspath{{pics/}} % где лежат картинки

\counterwithin{figure}{section}

% Title Page
\title
{
\hfill \break	\hfill \break
\hfill \break	\hfill \break
Лабораторная работа 4.7.3

ИЗУЧЕНИЕ ГОЛОГРАММЫ
}
\author{Андрей Державин, Б01-901}

%\setcounter{secnumdepth}{0}

\begin{document}

\maketitle


\thispagestyle{empty} % выключаем отображение номера для этой страницы

\newpage

\tableofcontents % Вывод содержания
\thispagestyle{plain}
\newpage


\paragraph{Цель работы:}

ознакомление с методами получения и анализа поляризованного света.

\paragraph{В работе используются:}

оптическая скамья с осветителем; зеленый светофильтр; два поляроида; черное зеркало; 
полированная эбонитовая пластинка; стопа стеклянных пластинок; слюдяные пластинки 
разной толщины; пластинки в $1/4$ и $1/2$ длины волны; пластинка в одну длину волны для 
зеленого света (пластинка чувствительного оттенка)

\section{Теоретические сведения}

  \subsection{Определение направления разрешённой плоскости колебаний поляроида}
  \import{src/}{1_1.tex}

  \subsection{Получение эллиптически поляризованного света}
  \import{src/}{1_2.tex}

  \subsection{Анализ эллиптически поляризованного света}
  \import{src/}{1_3.tex}

  \subsection{Пластинка чувствительного оттенка}
  \import{src/}{1_4.tex}

  \subsection{Интерференция поляризованных лучей}
  \import{src/}{1_5.tex}
\newpage
\section{Ход работы}
\import{src/}{hod_A.tex}

\end{document}