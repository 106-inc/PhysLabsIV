Определить направление разрешённых колебаний поляроида проще всего с помощью чёрного зеркала.
	
При падении света на отражающую поверхность под углом Брюстера, свет в отражённом луче почти полностью поляризован, а вектор E
параллелен отражающей поверхности ("<правило иголки">). Луч света,
прошедший поляроид и отразившийся от чёрного зеркала, имеет минимальную интенсивность при выполнении двух условий: во-первых, свет
падает на отражающую поверхность под углом Брюстера и, во-вторых,
в падающем пучке вектор E лежит в плоскости падения.

Вращая поляроид вокруг направления луча и чёрное зеркало вокруг
оси, перпендикулярной лучу, методом последовательных приближений
можно добиться минимальной яркости луча, отражённого от зеркала,
и таким образом определить разрешённое направление поляроида.

Измеряя угол поворота зеркала (угол Брюстера), нетрудно определить коэффициент преломления материала, из которого изготовлено
зеркало. Описанный метод часто используется для измерения коэффициента преломления непрозрачных диэлектриков.