\begin{wrapfigure}{r}{0.35\linewidth} 
	\includegraphics[width=\linewidth]{1}
	\caption{Разложение линейно поляризованного света по главным направлениям двоякопреломляющей пластинки}
	\label{ris 1}
\end{wrapfigure}

Эллиптически поляризованный свет можно получить из линейно поляризованного с
помощью двоякопреломляющих кристаллических пластинок.

Двоякопреломляющая пластинка имеет два взаимно перпендикулярных главных направления, совпадающих с осями эллипсоида диэлектрической проницаемости. Волны, поляризованные вдоль главных направлений, распространяются в пластинке с разными скоростями, не изменяя характера своей поляризации. Эти волны называются главными. Мы будем обозначать показатели преломления для главных волн через $ n_x $ и $ n_y $, где $ x $ и $ y $ --- главные направления кристаллической пластинки (рис. 1).

Пусть на пластинку падает линейно поляризованная волна, электрический вектор которой ориентирован под некоторым углом $ \alpha $ к оси
$ x $. Разложим вектор $ \mathbf{E} $ на составляющие $ E_x $ и $ E_y $. На входе пластинки $ E_x $ и $ E_y $ находятся в фазе. На выходе из-за разности скоростей между ними появляется разность хода $ d(n_x - n_y) $, при этом сдвиг фаз определяется соотношением
\begin{equation}\label{}
\Delta \phi =  \dfrac{2\pi}{m} = k d(n_x - n_y)
\end{equation}
Как уже отмечалось, при сложении двух взаимно перпендикулярных колебаний, обладающих некоторым сдвигом фаз, образуется колебание, поляризованное по эллипсу.

\begin{figure}[h!]
  \center{
	\includegraphics[width=0.5\linewidth]{2}}
	\caption{Поворот направления колебаний с помощью пластинки в $ \lambda / 2 $}
	\label{ris 2}
\end{figure}

Рассмотрим практически важные частные случаи.
\begin{enumerate}
 		
  \item Пластинка даёт сдвиг фаз $ 2\pi $ (пластинка в длину волны $ \lambda $). В результате сложения волн на выходе пластинки образует-
ся линейно поляризованная волна с тем же направлением колебаний, что и в падающей волне.

 \item Пластинка даёт сдвиг фаз $ \pi $ (пластинка в полдлины волны $ \lambda / 2 $). На выходе пластинки снова образуется линейно поляризованная волна. Направление $ bb' $ колебаний этой волны повёрнуто относительно направления $ aa' $ колебаний падающей волны (рис. 2). Как нетрудно сообразить, направление $ bb' $ является зеркальным отображением направления $ aa' $ относительно одного из главных направлений пластинки. Такую пластинку используют для поворота направления колебаний линейно поляризованного света.
 
 \item Пластинка создаёт между колебаниями сдвиг фаз $ \pi/2 $ (пластинка
в четверть длины волны). При сложении двух взаимно перпендикулярных колебаний, имеющих разность фаз $ \pi/2 $, образуется эллипс, главные оси которого совпадают с координатными осями $ x $ и $ y $. При равенстве амплитуд возникает круговая поляризация.
  
\end{enumerate}

Следует отметить, что, говоря о пластинках $ \lambda , \lambda/2, \lambda/4  $ и т. д., всегда подразумевают какую-либо вполне определённую монохроматическую
компоненту (например, пластинка $ \lambda/2 $ для зелёного света). Если на двоякопреломляющую пластинку падает не монохроматический свет, то на
выходе из неё для разных спектральных компонент эллипсы поляризации будут различными.
