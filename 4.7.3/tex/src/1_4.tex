\begin{wrapfigure}{l}{0.35\linewidth}
	\includegraphics[width=\linewidth]{3}
	\caption{Пластинка
чувствительного
оттенка}
	\label{ris 3}
\end{wrapfigure}

Выше предполагалось известным, какому из двух главных направлений пластинки в четверть длины волны соответствует большая скорость распространения света.
Установить это можно различными способами, например с помощью
пластинки чувствительного оттенка (так называют пластинку в $ \lambda $
для зелёной спектральной компоненты, $ \lambda = 560 $ нм).

Пластинка имеет форму стрелы (рис. 3), вдоль оси которой расположено главное направление, соответствующее большей скорости распространения.

Если пластинка чувствительного оттенка помещена между скрещенными поляроидами и главные направления пластинки не параллельны
направлениям разрешённых колебаний поляроидов, то при освещении
белым светом пластинка кажется окрашенной в лилово-красный цвет.
Это объясняется тем, что зелёная компонента линейно поляризованного света при прохождении пластинки не меняет поляризации и задерживается вторым поляроидом. Для красной и фиолетовой компонент
пластинка создаёт сдвиг фаз, несколько отличный от $ 2\pi $. На выходе
из пластинки красная и фиолетовая компоненты оказываются поэтому
эллиптически поляризованными и частично проходят через второй поляроид. Таким образом, в известном смысле наблюдаемый в указанном
опыте цвет пластинки дополнителен к зелёному.

Если между скрещенными поляроидами поместить пластинку чувствительного оттенка
($ \lambda $) и пластинку в $ \lambda/4 $ так, чтобы их главные
направления совпадали, цвет пластинки изменится. Если у пластинки чувствительного оттенка и пластинки в $ \lambda/4  $совпадут главные направления, соответствующие большей скорости распространения, то разность хода между $ E_x $ и $ E_y $ для зелёного света составит уже $ 5\lambda/4 $. Это соответствует разности хода в $ \lambda $ для света с большей длиной волны, т. е. для "<более красного"> света. При освещении
этих пластинок (напомним, что они расположены между скрещенными поляроидами) белым светом теперь погасится не зелёная, а красная
часть спектра, и проходящий свет будет казаться зеленовато-голубым.
Если же главные направления, соответствующие большей скорости распространения, у пластинки чувствительного оттенка и у пластинки
в $ \lambda/4 $ окажутся перпендикулярными, то проходящий свет приобретёт
оранжево-желтую окраску (погасится фиолетово-голубая часть спектра).

Изменение цвета позволяет, таким образом, определить, какое из
главных направлений пластинки в $ \lambda/4 $ соответствует большей скорости
распространения.
